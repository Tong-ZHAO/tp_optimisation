%%%%%%%%%%%%%%%%%%%%%%%%%%%%%%%%%%%%%%%%%
% University Assignment Title Page 
% LaTeX Template
% Version 1.0 (27/12/12)
%
% This template has been downloaded from:
% http://www.LaTeXTemplates.com
%
% Original author:
% WikiBooks (http://en.wikibooks.org/wiki/LaTeX/Title_Creation)
%
% License:
% CC BY-NC-SA 3.0 (http://creativecommons.org/licenses/by-nc-sa/3.0/)
% 
% Instructions for using this template:x
% This title page is capable of being compiled as is. This is not useful for 
% including it in another document. To do this, you have two options: 
%
% 1) Copy/paste everything between \begin{document} and \end{document} 
% starting at \begin{titlepage} and paste this into another LaTeX file where you 
% want your title page.
% OR
% 2) Remove everything outside the \begin{titlepage} and \end{titlepage} and 
% move this file to the same directory as the LaTeX file you wish to add it to. 
% Then add \input{./title_page_1.tex} to your LaTeX file where you want your
% title page.
%
%%%%%%%%%%%%%%%%%%%%%%%%%%%%%%%%%%%%%%%%%
%\title{Title page with logo}
%----------------------------------------------------------------------------------------
%   PACKAGES AND OTHER DOCUMENT CONFIGURATIONS
%----------------------------------------------------------------------------------------

\documentclass[10pt]{article}
\usepackage{graphicx}
\usepackage[colorinlistoftodos]{todonotes}

\usepackage[utf8]{inputenc}
% Use times if you have the font installed; otherwise, comment out the
% following line.
%\usepackage[latin]{inputenc}
\usepackage[english]{babel}
\usepackage[T1]{fontenc}
\usepackage{tgtermes}
\usepackage{float, caption}

\usepackage{amsfonts}
\usepackage{amsmath}
%\usepackage{bbold}
%\usepackage{multirow}
\usepackage{amsthm,amssymb}
\renewcommand{\qedsymbol}{$\blacksquare$}
\newcommand\independent{\protect\mathpalette{\protect\independenT}{\perp}}
\def\independenT#1#2{\mathrel{\rlap{$#1#2$}\mkern2mu{#1#2}}}
\usepackage{graphicx} % Required for including images
\graphicspath{{figures/}} % Directory in which figures are stored

\usepackage{xspace}
\usepackage{array}
\usepackage{graphicx}
\usepackage{latexsym}
\usepackage{mathtools}
\usepackage{listings}
\usepackage{times}
\usepackage{indentfirst}
\usepackage{algorithm}% http://ctan.org/pkg/algorithms
\usepackage[noend]{algpseudocode}% http://ctan.org/pkg/algorithmicx
\renewcommand{\algorithmicrequire}{\textbf{Input  :}}  
\renewcommand{\algorithmicensure}{\textbf{Output:}}
\renewcommand{\baselinestretch}{1.6}

% The preamble here sets up a lot of new/revised commands and
% environments.  It's annoying, but please do *not* try to strip these
% out into a separate .sty file (which could lead to the loss of some
% information when we convert the file to other formats).  Instead, keep
% them in the preamble of your main LaTeX source file.


% The following parameters seem to provide a reasonable page setup.

\topmargin 0.0cm
\oddsidemargin 0.15cm
\textwidth 16cm 
\textheight 22cm
\footskip 1.0cm
\setlength{\parindent}{2em}


%The next command sets up an environment for the abstract to your paper.

\newenvironment{sciabstract}{%
	\begin{quote} \bf}
	{\end{quote}}


% If your reference list includes text notes as well as references,
% include the following line; otherwise, comment it out.

\renewcommand\refname{References and Notes}

% The following lines set up an environment for the last note in the
% reference list, which commonly includes acknowledgments of funding,
% help, etc.  It's intended for users of BibTeX or the {thebibliography}
% environment.  Users who are hand-coding their references at the end
% using a list environment such as {enumerate} can simply add another
% item at the end, and it will be numbered automatically.

\newcounter{lastnote}
\newenvironment{scilastnote}{%
\setcounter{lastnote}{\value{enumiv}}%
\addtocounter{lastnote}{+1}%
\begin{list}%
{\arabic{lastnote}.}
{\setlength{\leftmargin}{.15in}}
{\setlength{\labelsep}{.5em}}}
{\end{list}}

\begin{document}

\begin{titlepage}

\newcommand{\HRule}{\rule{\linewidth}{0.5mm}} % Defines a new command for the horizontal lines, change thickness here

\center % Center everything on the page
 
%----------------------------------------------------------------------------------------
%   HEADING SECTIONS
%----------------------------------------------------------------------------------------

\textsc{\LARGE Ecole Nationale des Ponts et Chaussées}\\[1.5cm] % Name of your university/college
\textsc{\Large Optimisation et Contrôle}\\[0.5cm] % Major heading such as course name


%----------------------------------------------------------------------------------------
%   TITLE SECTION
%----------------------------------------------------------------------------------------

\HRule \\[0.4cm]
{ \huge \bfseries Projet sur les reseaux de distribution d'eau}\\[0.4cm] % Title of your document
\HRule \\[1.5cm]
 
%----------------------------------------------------------------------------------------
%   AUTHOR SECTION
%----------------------------------------------------------------------------------------

\begin{center}

Author: \emph{\\ \vspace{0.8em} Matthieu TOULEMENT \\ Tong ZHAO}

\end{center}


% If you don't want a supervisor, uncomment the two lines below and remove the section above
%\Large \emph{Author:}\\
%John \textsc{Smith}\\[3cm] % Your name

%----------------------------------------------------------------------------------------
%   DATE SECTION
%----------------------------------------------------------------------------------------

{\large \today}\\[2cm] % Date, change the \today to a set date if you want to be precise

%----------------------------------------------------------------------------------------
%   LOGO SECTION
%----------------------------------------------------------------------------------------

\includegraphics[width=10em]{header_logo.png}\\[1cm] % Include a department/university logo - this will require the graphicx package
 
%----------------------------------------------------------------------------------------

\vfill % Fill the rest of the page with whitespace

\end{titlepage}

\section{Présentation du problème}

On se concentre sur le problème de la résolution des équations décrivant l'état d'équilibre d'un réseau de distribution d'eau potable au cours de ce projet. 

\section{Modélisation du problème}

\section{Séance de TP No 1}

Dans cette partie, on s'intéresse au problème primal d'optimisation sans contrainte.

\vspace{-1.5em}
\begin{align}
  min_{q_C \in \mathbb{R}^{n-m_d}} \frac{1}{3} \left \langle q^{(0)} + B q_c, r \cdot (q^{(0)} + B q_c) \cdot |q^{(0)} + B q_c| \right \rangle + \left \langle p_r, A_r(q^{(0)} + B q_c) \right \rangle
\end{align}

\subsection{Le calcul du gradient}

D'abord, on calcule le gradient du premier terme.

\vspace{-1.5em}
\begin{align}
  f_1 (q_c) = \left \langle q^{(0)} + B q_c, r \cdot (q^{(0)} + B q_c) \cdot |q^{(0)} + B q_c| \right \rangle
\end{align}

On pose $q = q^{(0)} + B q_c$ et on calcule

\vspace{-1.5em}
\begin{align}
  \cfrac{\partial f_1}{\partial q} &= \cfrac{\partial  \left \langle q, r \cdot q \cdot |q| \right \rangle}{\partial q} = \cfrac{\partial(r \cdot q \cdot q \cdot |q|)}{\partial q} = 3sign(q) \cdot r \cdot q \cdot q = 3r \cdot q \cdot |q|
\end{align}

Ensuite, on calcule le gradient du second terme.

\vspace{-1.5em}
\begin{align}
  f_2 (q_c) = \left \langle p_r, A_r (q^{(0)} + B q_c) \right \rangle
\end{align}

\vspace{-1.5em}
\begin{align}
  \cfrac{\partial f_2}{\partial q} = \cfrac{\partial (p_r \cdot A_r q)}{\partial q} = A_r^T p_r
\end{align}

Sachant que $\cfrac{\partial q}{\partial q_c} = B^T$, on calcule le gradient de la fonction $F$.

\vspace{-1.5em}
\begin{align}
  G(q_c) = \cfrac{\partial F}{\partial q_c} = \cfrac{\partial (3f_1 + f_2)}{\partial q} \cfrac{\partial q}{\partial q_c} = B^T (r \cdot q \cdot |q| + A_r^T p_r)
\end{align}

\subsection{Le calcul du hessien}

On veut calculer:

\vspace{-1.5em}
\begin{align}
  H(q_c) = \cfrac{\partial G}{\partial q_c} = \cfrac{\partial (B^T(r \cdot q \cdot |q| + A_r^T p_r))}{\partial q_c}
\end{align}

On pose $d = r \cdot q \cdot |q|$, alors

\vspace{-1.5em}
\begin{align}
 H(q_c) = \cfrac{\partial (B^T d)}{\partial d} \cfrac{\partial d}{\partial q} \cfrac{\partial q}{\partial q_c} 
\end{align}

Sachant que:

\vspace{-1.5em}
\begin{align*}
  \frac{\partial d}{\partial q} &=
  \begin{bmatrix}
    \frac{\partial d_1}{\partial q_1} & \cdots & \frac{\partial d_1}{\partial q_n} \\
    \vdots & \ddots & \vdots \\
    \frac{\partial d_n}{\partial q_1} & \cdots & \frac{\partial d_n}{\partial q_n}
  \end{bmatrix} \mbox{\ \ \ d'où \ \ } \frac{\partial d_i}{\partial q_j} =
  \begin{cases} 
    2r_i|q_i|,  & i = j \\
    0, & i \ne j
  \end{cases}
\end{align*}

On a:

\vspace{-1.5em}
\begin{align}
  \frac{\partial d}{\partial q} = diag((2r_i|q_i|)_{i=1 \cdots n})
\end{align}

On en déduit que:

\vspace{-1.5em}
\begin{align}
 H(q_c) = 2B^T diag((2r_i|q_i|)_{i=1 \cdots n}) B 
\end{align}


\end{document}

              